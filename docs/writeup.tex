%%%%%%%%%%%%%%%%%%%%%%%%%%%%%%%%%%%%%%%%%%%%%%%%%%%%%%%%%%%%%%%%%%%%%%%%%%%%%%%%
%2345678901234567890123456789012345678901234567890123456789012345678901234567890
%        1         2         3         4         5         6         7         8

\documentclass[letterpaper, 10 pt, conference]{ieeeconf}  % Comment this line out
                                                          % if you need a4paper
%\documentclass[a4paper, 10pt, conference]{ieeeconf}      % Use this line for a4
                                                          % paper

\IEEEoverridecommandlockouts                              % This command is only
                                                          % needed if you want to
                                                          % use the \thanks command
\overrideIEEEmargins
% See the \addtolength command later in the file to balance the column lengths
% on the last page of the document



% The following packages can be found on http:\\www.ctan.org
%\usepackage{graphics} % for pdf, bitmapped graphics files
%\usepackage{epsfig} % for postscript graphics files
%\usepackage{mathptmx} % assumes new font selection scheme installed
%\usepackage{times} % assumes new font selection scheme installed
%\usepackage{amsmath} % assumes amsmath package installed
%\usepackage{amssymb}  % assumes amsmath package installed

\title{\LARGE \bf
Evaluating Style Transfer
}
\author{Peter W. Kong}
%\author{ \parbox{3 in}{\centering Huibert Kwakernaak*
%         \thanks{*Use the $\backslash$thanks command to put information here}\\
%         Faculty of Electrical Engineering, Mathematics and Computer Science\\
%         University of Twente\\
%         7500 AE Enschede, The Netherlands\\
%         {\tt\small h.kwakernaak@autsubmit.com}}
%         \hspace*{ 0.5 in}
%         \parbox{3 in}{ \centering Pradeep Misra**
%         \thanks{**The footnote marks may be inserted manually}\\
%        Department of Electrical Engineering \\
%         Wright State University\\
%         Dayton, OH 45435, USA\\
%         {\tt\small pmisra@cs.wright.edu}}
%}

% \author{Huibert Kwakernaak$^{1}$ and Pradeep Misra$^{2}$% <-this % stops a space
% \thanks{*This work was not supported by any organization}% <-this % stops a space
% \thanks{$^{1}$H. Kwakernaak is with Faculty of Electrical Engineering, Mathematics and Computer Science,
%         University of Twente, 7500 AE Enschede, The Netherlands
%         {\tt\small h.kwakernaak at papercept.net}}%
% \thanks{$^{2}$P. Misra is with the Department of Electrical Engineering, Wright State University,
%         Dayton, OH 45435, USA
%         {\tt\small p.misra at ieee.org}}%
% }


\begin{document}



\maketitle
\thispagestyle{empty}
\pagestyle{empty}


%%%%%%%%%%%%%%%%%%%%%%%%%%%%%%%%%%%%%%%%%%%%%%%%%%%%%%%%%%%%%%%%%%%%%%%%%%%%%%%%
\begin{abstract}

This project formalizes the problem of evaluation of style transfer and explores improves on the state of the art metrics.

\end{abstract}


%%%%%%%%%%%%%%%%%%%%%%%%%%%%%%%%%%%%%%%%%%%%%%%%%%%%%%%%%%%%%%%%%%%%%%%%%%%%%%%%


\section{Introduction}
                
        
Style identification is the identification of style in a text passage by an automated process, often a language model. Generative models can generate text in one or more styles, given content and style parameters. Style identification has proven utility in applications like author identification (Houvardas). Yet, while much research has focused on semantic transfer from one human language to another, there has been little attention paid to intra-lingual style transfer. [Intra-lingual] style transfer affords many interesting applications: authorship obfuscation, text simplification, and style enhancement.

The existing literature on style transfer uses a combination of BLEU scores, perplexity [Ficler], black box neural classifiers, and manual labeling to evaluation the quality of a transfer. Each of these has serious problems (see SECTION)


This paper’s objectives are three-fold:
\begin{itemize}
    \item Identify characteristics desirable in a style transfer evaluation metric
    \item Formalize the evaluation objective
    \item Explore a specific approach and discuss results
\end{itemize}
          

          
\section{Related work}

There is active research in style transfer. Ficler uses fixed, human- intelligable stylistic parameters like ’length’ and ’sentiment’, which can be extracted from specific training data with heuristics. They use perplexity as an evaluation metric.
\\ 

[Fu, et al.] addresses the problem of learning style transfer without the use of parallel corpora, using style embeddings (a previous work) and an auto-encoder model with multiple decoders. Attempts are made to improve evaluation metrics, but they do not appear to improve on human judgment and black-box classifiers. Example outputs from models are hard to visualize/express in a relatable way.
\\ 

[Hu] seeks to generate realistic (English) sentences using a VAE in a wake-sleep configuration. Semantic and stylistic signals can be injected to coerce specific meaning and styles.
Outperforms stated previous work: a semi-supervised VAE (Kingma 2014). Reaches about 83pct sentiment classificiation accuracy, although performance varies greatly with datasets used. 
Metric: label accuracy. Method: External sentiment clasifiers.
\\ 

[Prabhumoye]'s proposed model creates a style-agnostic representation of an English input sentence using A->B then B->A neural machine translators, with the objective of style transfer with minimal semantic drift. A style-specific transformation step adds specific style to the style-agnostic representation. This paper confirms that BLEU and ROUGE metrics are not suitable for capturing meaning.

\section{Criticism of state of the art metrics}
[TO EXPAND ON]
\begin{itemize}
  \item Terrible accuracy (perplexity
  \item Easily fooled (bleu)
  \item Low explainability (neural models)
  \item Intractible (manual)
\end{itemize}

\section{Formalizing the problem}
  \subsection{problem formalization} We formalize the style transfer problem as:
    Given:
    \begin{itemize}
      \item a style transfer model $M_{AB}$ that attempts to transfer an input sentence in style A to style B
      \item a candidate output sentence from $M_{AB}$:  $x_{AB}^i$
      \item a reference sentence and score pair $R_{AB}^i ; y_i$ that is scored in a range of $[0,1]$ by a trusted external source
    \end{itemize}

  The evaluation metric E($x_{AB}^i$, $R_{AB}^i, y_i$) will output a score. A successful metric E will output a score $\hat{y}$, where $|y-\hat{y}|$ is minimized.
  \subsection{Desired Evaluation metric traits}
    \begin{itemize}
      [TO EXPAND ON]
      \item intuitive
      \item linear in the input to $E()$
      \item relies only on $X$, $R$, $Y$
      \item if it evaluates content in addition to style, it should do so independently
    \end{itemize}
  \subsection{discussion of available paths} 
  neural: pros and cons\\
  human engineered features: pros and cons\\
  \subsection{Chosen path}
  human engineered features: include reasoning
  
\section{Experiment setup}
\subsection{Data}

All aligned sentences from the Shakespeare dataset were used, from the following plays: 'othello', 'antony-and-cleopatra', 'asyoulikeit', 
                     'errors', 'hamlet', 'henryv', 'juliuscaesar', 'lear', 'macbeth', 
                     'merchant', 'msnd', 'muchado', 'richardiii', 'romeojuliet', 
                     'shrew', 'tempest', 'twelfthnight'

42,000 parallel sentences: original and sparknotes version.

The topics encompass a wide variety of content and sentiment: comedies, tragedies, histories, although it is important to note that they are all written in the screenplay format. While this is an amazing corpus to work with, there are some natural limitations.

Here are some examples of parallel aligned sentences. As you can see, differences in percieved style can vary significantly from passage to passage (original style on the left):


\begin{table}[h]
  \caption{Aligned Text Examples}
  \label{table_example}
  \begin{center}
    \begin{tabular}{| p{3.5cm}  | p{3.5cm} |}
    \hline
    What's the matter, lieutenant? & What's the matter, lieutenant?\\
    \hline
    Tell me this, I pray: Where have you left the money that I gave you? & Answer me this, please: where's the money I gave you?\\
    \hline
    Lucius, who's that knocking? & Lucius, who's that knocking?\\
    \hline
    Gentlemen, forward to the bridal dinner. & Gentlemen, on to the bridal dinner.\\
    \hline
    \end{tabular}
  \end{center}
\end{table}




The original sentences were automatically assigned a "class 1" label, and the sparknotes sentences a "class 2" label, this generating a classical supervised learning dataset with "gold" labels.

This dataset was randomly shuffled. 70pct of corpus was designated the train set, 20pct the validation set, and 10pct the holdout test set.
 
We experimented with a variety of models. However, the choice of experiments was guided by our evaluator design ideals [see prev section].

[spacy POS tagger]. fortunately quite good, even on the relatively ancient shakespearean corpus (train on default web corpora distribution). example:

Go bid the priests do present sacrifice And bring me their opinions of success\\

Go VERB    
bid VERB   
the DET   
priests NOUN 
do  VERB   
present ADJ  
sacrifice
And CCONJ 
bring VERB  
me  PRON   
their ADJ    
opinions  NOUN  
of  ADP    
success NOUN 


\subsection{Models}
That had linear predictive power, as opposed to binary classification. 

Models employed:
logistic regression\\
SVM




\subsection{feature engineering}
Features created:
guided by intuition, Strunk and white

We used grid search to find successful combinations of engineered features.
  
\section{Results}
  include examples of type 1 and type 2 errors\\
  include accuracy of best model\\
  include statement of features used.\\
\section{Analysis}
  since this is an evaluation metric, it's important to strive for a non-binary, classifier-type metric so that the metric will yield maximal information about input.
  \subsection{Corpora}
    Parallel, sentence aligned Shakespeare dataset. original + cliffnotes. [Coco xu]
    Analysis of bleu score:\\

    style1 vs style2 averaged bleu: 44 \\
randomized:.0175 \\

  BLEU scores range from 0 to 1. We can see from the differences of intra-style vs inter-style blue scores that BLEU certainly contains some signal. Unfortunately, since our labels of styles are binary, it is difficult to investigate this further and determine whether or not inter-style BLEU scores correlate smoothly/monotonically with inter-style change.

  \subsection{Error analysis}
      softmax\\
      f1 score not sensible for multiclass, dynamic class problem like this.\\
      accuracy would provide some judgment of a hypothetical model, but we want weightings/confidence levels signalled as well.
  \subsection{Acknowledged shortcomings}
  assumption of evenly distributed class (style) priors
One dataset, where part of the stated task of style B was shortening.

\section{Future work}
expand approach to handle passages larger than sentences\\
What is the class prior for a style? Can we pick a meaningful value?\\
How do we deal with fluidity and overlapping between styles, or shifting style boundaries?\\
Style-author hypothesis:
  That an author always writes in the same style


...
\section{Conclusions}



\addtolength{\textheight}{-12cm}   % This command serves to balance the column lengths
                                  % on the last page of the document manually. It shortens
                                  % the textheight of the last page by a suitable amount.
                                  % This command does not take effect until the next page
                                  % so it should come on the page before the last. Make
                                  % sure that you do not shorten the textheight too much.

%%%%%%%%%%%%%%%%%%%%%%%%%%%%%%%%%%%%%%%%%%%%%%%%%%%%%%%%%%%%%%%%%%%%%%%%%%%%%%%%



%%%%%%%%%%%%%%%%%%%%%%%%%%%%%%%%%%%%%%%%%%%%%%%%%%%%%%%%%%%%%%%%%%%%%%%%%%%%%%%%



%%%%%%%%%%%%%%%%%%%%%%%%%%%%%%%%%%%%%%%%%%%%%%%%%%%%%%%%%%%%%%%%%%%%%%%%%%%%%%%%
\section*{APPENDIX}

...



\begin{thebibliography}{99}

\bibitem{c1} example of bib item 

\end{thebibliography}

\end{document}
